\section{Conclusion}

L'utilisation d'outils tel que "MagicDraw" nous a permis de se rendre compte que chaque étapes
de la réalisation d'un produit sont importantes et ne doivent pas être négligées. De la définition
des besoins avec les parties prenantes jusqu'à la livraison du produit des outils de modélisation
permettent aux concepteurs d'avancer étapes par étapes avec les parties prenantes en garantissant
de remplir toutes les éxigences fixées depuis le début.\\

La principale difficulté aura été de savoir bien fixer le niveau d'abstraction et de détail de chaque
exigences pour éviter d'entre trop dans les détails ce qui n'été pas le but recherché pour ce travail
de groupe. Le fait d'effectuer aussi ce travail en groupe permet d'avancer plus rapidement sur des
projets multidisciplinaires en ayant différents points de vues.\\

Des améliorations du système peuvent être aussi envisagées comme passer sur un niveau de tension plus 
élevé (batterie de 24V) en adaptant certains systèmes comme la motorisation ce qui garantira une 
autonomie plus élevée du Robot. Une intégration plus optimale du système peut être aussi révisé, une 
gestion autonome des batteries ainsi que des plants peuvent être de futurs améliorations du Robot.\\

Au fil de ce projet nous avons pris concience de l'importance d'une bonne définitions
des limites de notre système. A partir d'une meme liste d'exigences, beaucoup d'idée 
d'organisation ainsi que de solution émerges. L'importance des réunions pour discuter
des différents point ainsi qu'une bonne utilisation des outils de modélisations universel est primordiale pour
le bon déroulement d'un projet. De plus cette méthode de travail offre un lisibilité claire et précise du système.

