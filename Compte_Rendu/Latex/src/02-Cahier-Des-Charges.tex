\section{Cahier des charges du Projet "Barre-Franche"}
\subsection{Introduction technique}
Le projet qui nous est demandé est basé sur les SOC et plus particulièrement sur un FPGA, de chez Altera, embarqué dans un voilier pour en piloter la barre-franche.
\newline
Ce dispositif électronique fonctionnera à l'aide de différentes entrées/sorties (gyroscope, anémométre, GPS, convertisseur analogique/numérique, vérin, boutons, buzzer).
\subsection{Cahier des charges général}
Durant ce projet, nous allons utiliser les différentes compétences acquises à travers les différents cours de l'année et les mettre en corélation pour mener à bien ce dernier. Le projet devra respecter certains critères présentés ci-dessous :
\begin{enumerate}
    \item   Le dispositif devra utiliser un appareil de mesure pour capter la valeur de la vitesse du vent.
    \item   Le dispositif devra utiliser un appareil de mesure pour capter la direction du vent.
    \item   Le dispositif devra receptionner des données GPS, traiter ces données et agir sur le dispositif en fonction des résultats obtenus après traitement.
    \item
\end{enumerate}
\subsection{Cahier des charges technique}


